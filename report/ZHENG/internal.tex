\subsection{クラス図}
	このクラス図はチームメンバーと共同に作ったものである。
	
	\insertfigure{11}{diagram.eps}
\subsection{各クラスの説明}
	\begin{itemize}
		\item game:
		
		ゲームの運行を制御し,全体モジュール(コント ロールの入力、ゲームの表現、オブジェクトの管理、ステージの管理、 スコアの計算)のベースとして総合クラスである。
		
		\item objectList:
		
		unitの種類(enemy, supply, bullet etc.)により保存用配列として実装したオブジェクトリストクラスである。
		
		\item stage:
		
		段階的に敵戦機の種類とか速度とか再生時間とか設計し再生関数を呼び出し用のマネージャークラスである。
		
		\item unit:
		
		全ての独立単位オブジェクトのプロトタイプクラスである。幅と長さ、座標、スピードなど基本の属性初始化し,公用動作メソッドも定義する。
		
		\item player:
		
		プレーヤー戦機の抽象クラスである。unitから継承し,その上に特有なライフ値と火力に関する属性を追加する
		
		\item enemy:
		
		敵戦機の抽象クラスである。unitから継承し,その上に特有なライフ値と火力と戦機種類に関する属性を追加する
			\begin{itemize}
				\item enemyA:普通の敵戦機のクラスである。
				\item enemyB:キャプテンの敵戦機のクラスである。
				\item enemyC:ボス敵戦機のクラスである。
			\end{itemize}
			
		\item supply:
		
		敵を倒れた時出る供給である。二つ種類で,ライフ値か火力の一つサポートする。
		
		\item bullet:
		
		プレーヤー戦機から発射される弾である。
		
		\item enemyBullet:
		
		敵戦機から発射される弾である。
		
	\end{itemize}