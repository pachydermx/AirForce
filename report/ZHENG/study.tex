	\subsection{プログラム上}
		\begin{itemize}
			\item JavaScriptのプロトタイプチェンによるオブジェクト指向プログラミングで実装することにより、テストの便利性も考え、プログラムの各モージュルを独立させた。
			\item 共有するメソッドをプロトタイププロパティーにアサインすることにより、無駄なメモリ使用を削除し、良い性能を実現できた。
			\item オブジェクトの共通部分を継承ツリーとして作成したため、将来新しい敵、武器などを拡張しようとすると簡単にできる。
		\end{itemize}
	\subsection{デザイン上}
		\begin{itemize}
			\item GIMPを使いオリジナルな8bitスタイル画像素材を幾つか作成した。
			\item ゲームプレーの体験を改善する止めに,難しさをよく考え,段階的なシーンを設計する
			\item プレーヤーの武器の5段階のレベルを用意した。
			\item 敵を3種類を作り、それぞれ異なるヘルス、武器、速度などを設定した。
		\end{itemize}
	\subsection{チームワーク}
		\begin{itemize}
			\item GitHubを利用し、うまくチーム単位でのプログラミングができた。一方マルチスレッドの開発もできる。
			\item 開発中,ピアレビューしてエラーを発見し,ソフトのロバスト性を改善する。
		\end{itemize}