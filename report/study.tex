\subsection{工夫した点}
	\subsubsection{プログラム上}
		\begin{itemize}
			\item JavaScriptのプロトタイプチェンによるオブジェクト指向プログラミングで実装することにより、プログラムの各モージュルを独立させた。
			\item 共有するメソッドをプロトタイププロパティーにアサインすることにより、良い性能を実現できた。
			\item オブジェクトの共通部分を継承ツリーとして作成したため、将来新しい敵、武器などを拡張しようとすると簡単にできる。
		\end{itemize}
	\subsubsection{機能上}
		\begin{itemize}
			\item GIMPを使いオリジナルな画像素材を幾つか作成した。
			\item プレーヤーの武器の5段階のレベルを用意した。
			\item 敵を3種類を作り、それぞれ異なるヘルス、武器、速度などを設定した。
		\end{itemize}
	\subsubsection{チームワーク}
		\begin{itemize}
			\item GitHubを利用し、うまくチーム単位でのプログラミングができた。
		\end{itemize}
\subsection{感想}
		今回のプロジェクトでは、かなり古いバージョンの開発環境を利用し、開発を行った。特にブラウザーが古いため、デバッグ用のツールが一切搭載していなかった。また、最近の新しい技術(HTML5, CSS3)なども対応していなかった。それゆえ、自分のデバイスで実行できても、実験環境で動作できるかどうか分からなかった、もしくは、プログラムを入れたらちゃんと動作しない場合、コンソルなどの開発者ツールがないため、どこに間違えたかわかりにくかった。だが、最終的に設計の通りに実験環境でもうまく動作したのは良かったと思われる。将来古いシステムでの開発はやはり自分のマシンに同じ環境を仮想マシンとして作っておいた方がいいと思われる。