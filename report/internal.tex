\subsection{クラス図}
	このクラス図はチームメンバーと共同に作ったものである。
	
	\insertfigure{12}{diagram.eps}
\subsection{各クラスの説明}
	\begin{itemize}
		\item Game
		
		Gameクラスはゲーム全般の管理の役割を担当する。具体的に、コントロールの入力、ゲームの表現、オブジェクトの管理、ステージの管理、スコアの計算を行っている。
		\item unit
		
		unitクラスは、ゲームにある全てのオブジェクトの抽象クラスである。unitクラスはオブジェクトとしての基本機能を提供する。具体的には、オブジェクトの速度、位置、HP、状態、DOMオブジェクト、フレームごとにの行動などを実装した。
		\item Player
		
		Playerクラスは、unitクラスを拡張し、プレーヤーの機能を実装したものである。武器レベルはこのクラスに実装した。
		\item Enemy
		
		Enemyクラスは、全ての敵の抽象クラスである。Enemyクラスはunitクラスを拡張し、敵の機能を実装したものである。
		
		\item EnemyA, EnemyB, EnemyC
		
		EnemyA、EnemyB、EnemyCクラスは3種類の敵のクラスである。三つともEnemyクラスを拡張したものであり、それぞれ異なるHP、標準移動速度などを設置した。
		
		\item Bullet, EnemyBullet
		
		Bullet, EnemyBulletクラスはunitクラスを拡張し、弾の機能を実装したものである。
		
		\item ObjectList
		
		ObjectListは特殊な配列として実装したクラスである。配列に新しいエレメントを入れる時、そのエレメントに自動的にIDを割り合い、そのエレメントを簡単に削除できるようにした。
		
		\item Stage
		
		Stageはゲームのステージのマネージャークラスである。ステージの敵を管理する。
	\end{itemize}